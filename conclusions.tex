\chapter{Conclusions}
\label{chapter:conclusions}
\section{Summary}

A search for a Standard Model-like Higgs boson and anomalous neutral triple
gauge couplings has been conducted at $\sqrt{s} = 8~TeV$, using 19.6~\fbinv of
pp collision data. The analysis benefits greatly from having a clean signature,
allowing the usage of highly efficient reconstruction and selection algorithms
while maintaining a low level of reducible background. This background was
estimated using an entirely data-driven process, while the Standard Model ZZ
production and Higgs signal processes were simulated using Monte Carlo methods.

The search for a Standard Model Higgs boson in the $H\rightarrow ZZ\rightarrow
\ell\ell\ell\ell$ final state resulted in a discovery of a new boson, with a
6.1$\sigma$ excess at a hypothesized Higgs mass of 126~GeV. 

A search for the anomalous neutral triple gauge couplings \ffour and \ffive was
conducted. The results are fully consistent with Standard Model behaviour, and
95\% confidence level upper limits were set on the coupling strengths. The cross
section for Standard Model $ZZ\rightarrow\ell\ell\ell\ell$ production was
measured from the collected data, showing excellent agreement with theoretical
calculations.


\section{Future Outlook}
Looking toward the future, the properties of the new boson, the differential
cross sections of Standard Model ZZ production, and the continued search for
anomalous triple gauge couplings will all benefit from the increased cross
sections and luminosity expected from the LHC upgrades currently under way.
Increases in center-of-mass energies from 8~TeV to 13 or 14~TeV along with a
projected data collection of ${\sim500}$\fbinv by the end of the decade will
result in unprecedented precision in all ${(H\rightarrow)ZZ\rightarrow
\ell\ell\ell\ell}$ measurements.

The clean nature of the channel in addition to its fully reconstructed final
state allow an excellent handle on doing further analysis of the newly discovered
boson. An extension of the matrix element kinematic discriminant can be used to
comb out the likely spin properties, while creating a separate category for
events produced via Vector Boson Fusion (which includes a forward jet signature)
can be used to measure the relative fermionic and vector boson couplings. 
