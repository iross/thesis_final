\chapter{Introduction}
\label{chapter:introduction}

The theoretical model of particle physics today, known as the Standard Model, is
constructed to describe three fundamental forces between two classes of
particles. Despite the wide success of this theory, it is, at best, an
incomplete picture of the universe. Some of the largest scientific experiments
ever have been conducted with two goals in mind: to fill in the lingering
unknown pieces of the Standard Model and to explore for new physical phenomena.

This thesis describes the discovery of the Higgs boson, a long-missing
piece of the Standard Model, and a search for physics beyond the scope of the
Standard Model.  These searches were conducted using the Compact Muon Solenoid
(CMS) detector at the Large Hadron Collider (LHC), an enormous proton
accelerator beneath the French-Swiss border. The LHC is the highest-energy
particle accelerator ever built, allowing particle interactions to be probed at
an unrivaled scale.

We will begin with an overview of the Standard Model in Chapter
\ref{chapter:sm}, exploring the history of the theory, a general overview of
its foundations, and its shortcomings. Special attention is paid to the
electroweak symmetry breaking process known as the Higgs mechanism, due to its
relevance to this thesis and the field as it stands today. The state of the
Standard Model, Higgs physics, and physics beyond the Standard Model is
discussed in Chapter \ref{chapter:previous}.

We then move to the Large Hadron Collider and the Compact Muon Solenoid. Both
are described in detail in chapter \ref{chapter:expOverview}.  Then the process
of event generation and simulation is described in chapter
\ref{chapter:eventGenSim}.

The journey from detector basics to fully reconstructed and selected event is
chronicled in chapters \ref{chapter:eventReco} and \ref{chapter:analysis},
detailing the way that the detector components combine to deliver the
fundamental physics objects (chapter \ref{chapter:eventReco}, and how these
objects are utilized within an analysis flow (chapter \ref{chapter:analysis}).
Also described in this chapter are the methods for estimating the relevant
physics processes which act as background over the processes of interest,
estimates of systematic errors, and statistical treatment of the observations.

Finally, results are presented in chapter \ref{chapter:results}. Results for a
Higgs boson-like particle are presented, along with the measurement of ZZ
production. Limits are placed on the values of anomalous neutral triple gauge
coupling. These results are summarized in chapter \ref{chapter:conclusions}
before taking a quick glimpse at the results' place in the physics of today and
the near future.
